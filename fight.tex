\chapter{Aktionen im Kampf}
TODO

\section{Angriff}
\begin{itemize}
\item $W20 + Bonus + Proficiency >= AC$

\item $Hitpoints -= Schadensw\ddot{u}rfel + Modifikator$
\end{itemize}

Ein Angriff funktioniert in mehreren Schritten:

\begin{enumerate}

\item Angriffswurf: Der Angreifer würfelt einen 20-seitigen Würfel (W20) und addiert seinen Angriffsbonus. Der Angriffsbonus setzt sich aus dem relevanten Attributsmodifikator (z. B. Stärke für Nahkampfwaffen oder Geschicklichkeit für Finesse- oder Fernkampfwaffen) und dem Proficiency Bonus zusammen, falls der Charakter mit der Waffe geübt ist. Ein Würfelergebnis von 1 verfehlt das Ziel immer. Ein Würfelergebnis von 20 trifft immer \hyperref[sec:crit]{kritisch}.
\item Trefferwert vergleichen: Das Ergebnis der Angriffsrolle wird mit der Rüstungsklasse (Armor Class, AC) des Ziels verglichen. Wenn das Ergebnis gleich oder höher ist als die AC, trifft der Angriff.
\item Schadenswurf: Wenn der Angriff trifft, würfelt der Angreifer den Schaden, den die Waffe verursacht. Der Schadenswert setzt sich aus dem entsprechenden Schadenswürfel der Waffe und dem Modifikator des verwendeten Attributs zusammen (z. B. Stärke für Nahkampfangriffe).
\item Schaden anwenden: Der ermittelte Schaden wird von den Trefferpunkten (Hit Points) des Ziels abgezogen.
\end{enumerate}
Besondere Angriffe oder Fähigkeiten können zusätzliche Effekte oder Boni auf den Angriff oder Schaden gewähren.

\section{Initiative}
\begin{itemize}
\item $W20 + Geschicklichkeitsmodifikator + weitere Modifikatoren$
\end{itemize}

Initiative bestimmt die Reihenfolge, in der Charaktere und Gegner im Kampf handeln. Jeder würfelt einen 20-seitigen Würfel (W20) und addiert seinen Initiativebonus. Die Ergebnisse werden in absteigender Reihenfolge sortiert, und so wird die Reihenfolge der Aktionen festgelegt.

Der Initiativebonus ist der Wert, den ein Charakter zu seinem Wurf für die Initiative addiert. Er setzt sich aus dem Geschicklichkeitsmodifikator (Dexterity Modifier) und gegebenenfalls anderen Boni zusammen, die durch Fähigkeiten, Zauber oder Ausrüstung gewährt werden. Zum Beispiel, wenn ein Charakter einen Geschicklichkeitsmodifikator von +2 hat, dann ist sein Initiativebonus ebenfalls +2.

\section{Kritischer Schaden}
\label{sec:crit}
TODO